\documentclass[english,letter paper,12pt,leqno]{article}
\usepackage{stmaryrd}
\usepackage{amsmath, amscd, amssymb, mathrsfs, accents, amsfonts,amsthm}
\usepackage[all]{xy}
\usepackage{dsfont}
\usepackage{tikz}
\def\nicedashedcolourscheme{\shadedraw[top color=blue!22, bottom color=blue!22, draw=gray, dashed]}
\def\nicecolourscheme{\shadedraw[top color=blue!22, bottom color=blue!22, draw=white]}
\def\nicepalecolourscheme{\shadedraw[top color=blue!12, bottom color=blue!12, draw=white]}
\def\nicenocolourscheme{\shadedraw[top color=gray!2, bottom color=gray!25, draw=white]}
\def\nicereallynocolourscheme{\shadedraw[top color=white!2, bottom color=white!25, draw=white]}
\definecolor{Myblue}{rgb}{0,0,0.6}
\usepackage[a4paper,colorlinks,citecolor=Myblue,linkcolor=Myblue,urlcolor=Myblue,pdfpagemode=None]{hyperref}

\SelectTips{cm}{}

\setlength{\evensidemargin}{0.1in}
\setlength{\oddsidemargin}{0.1in}
\setlength{\textwidth}{6.3in}
\setlength{\topmargin}{0.0in}
\setlength{\textheight}{8.5in}
\setlength{\headheight}{0in}

\newtheorem{theorem}{Theorem}[section]
\newtheorem{proposition}[theorem]{Proposition}
\newtheorem{lemma}[theorem]{Lemma}
\newtheorem{corollary}[theorem]{Corollary}
\newtheorem{setup}[theorem]{Setup}

\newtheoremstyle{example}{\topsep}{\topsep}
	{}
	{}
	{\bfseries}
	{.}
	{2pt}
	{\thmname{#1}\thmnumber{ #2}\thmnote{ #3}}
	
	\theoremstyle{example}
	\newtheorem{definition}[theorem]{Definition}
	\newtheorem{example}[theorem]{Example}
	\newtheorem{remark}[theorem]{Remark}
	\newtheorem{strat}[theorem]{Strategy}

\numberwithin{equation}{section}

% Operators
\def\eval{\operatorname{ev}}
\def\res{\operatorname{Res}}
\def\Coker{\operatorname{Coker}}
\def\Ker{\operatorname{Ker}}
\def\im{\operatorname{Im}}
\def\can{\operatorname{can}}
\def\K{\mathbf{K}}
\def\D{\mathbf{D}}
\def\N{\mathbf{N}}
\def\LG{\mathcal{LG}}
\def\Ab{\operatorname{Ab}}
\def\stab{\operatorname{stab}}
\def\Hom{\operatorname{Hom}}
\def\modd{\operatorname{mod}}
\def\Modd{\operatorname{Mod}}
\def\be{\begin{equation}}
\def\ee{\end{equation}}
\def\nN{\mathds{N}}
\def\nZ{\mathds{Z}}
\def\nQ{\mathds{Q}}
\def\nR{\mathds{R}}
\def\nC{\mathds{C}}
\DeclareMathOperator{\Ext}{Ext}
\DeclareMathOperator{\Tr}{Tr}
\DeclareMathOperator{\End}{End}
\DeclareMathOperator{\rank}{rank}
\DeclareMathOperator{\tot}{Tot}
\DeclareMathOperator{\ch}{ch}
\DeclareMathOperator{\str}{str}
\DeclareMathOperator{\hmf}{hmf}
\DeclareMathOperator{\HMF}{HMF}
\DeclareMathOperator{\hf}{HF}
\DeclareMathOperator{\At}{At}
\DeclareMathOperator{\Cat}{Cat}
\DeclareMathOperator{\Spec}{Spec}

\begin{document}

% Commands
\def\Res{\res\!}
\newcommand{\ud}{\mathrm{d}}
\newcommand{\Ress}[1]{\res_{#1}\!}
\newcommand{\cat}[1]{\mathcal{#1}}
\newcommand{\lto}{\longrightarrow}
\newcommand{\xlto}[1]{\stackrel{#1}\lto}
\newcommand{\mf}[1]{\mathfrak{#1}}
\newcommand{\md}[1]{\mathscr{#1}}
\def\sus{\l}
\def\l{\,|\,}
\def\sgn{\textup{sgn}}

\title{$A_\infty$-minimal models of matrix factorisations}
\author{Daniel Murfet}

\maketitle

\begin{abstract}
We study $A_\infty$-minimal models of the differential graded algebras obtained from endomorphisms of generators of triangulated categories of matrix factorisations, giving explicit higher products for the cases of simple singularities. 
\end{abstract}

\section{Introduction}

Let $k$ be a characteristic zero field, and $W \in k[x_1,\ldots,x_n]$ a potential in the sense of \cite[\S 2.2]{lgdual} which has the origin as its only critical point. Choose a presentation of $W$ as a sum $W = x_1 W^1 + \cdots + x_n W^n$ with $W^i \in \mf{m}$. Then the generator of the homotopy category of matrix factorisations of $W$ is the stabilised diagonal
\begin{equation}\label{eq:kstab}
k^{\operatorname{stab}} = \Big( k[x] \otimes_k \bigwedge( k\psi_1 \oplus \cdots \oplus k \psi_n ), \;d_{k^{\stab}} = \sum_i x_i \psi_i^* + \sum_i W^i \psi_i \Big)\,.
\end{equation}
We calculate the $A_\infty$-minimal model of the $\nZ_2$-graded differential graded algebra
\be
\md{A}_W = \Big( \End_{k[x]}(k^{\operatorname{stab}}), \; \partial = [-, d_{k^{\stab}}] \; \Big)\,.
\ee 
The minimal model is a finite-dimensional $\nZ_2$-graded vector space $\md{M}_W$ with a family 
\[
\left\{ r_q: \big(\md{M}_W[1]\big)^{\otimes q} \lto \md{M}_W[1] \right\}_{q \ge 2}
\]
of odd $k$-linear maps satisfying the forward suspended $A_\infty$-constraints \cite{lazaroiu}. 

\section{Perturbation and Koszul complexes}

Set $R =  k[x_1,\ldots,x_n]$. To apply the $A_\infty$-minimal model construction to the DG-algebra $\End_R(k^{\stab})$, we would usually begin by finding a $k$-linear homotopy retract of this complex onto its cohomology. However, for our purposes it is better to first enlarge the DG-algebra to $S \otimes_k \End_R(k^{\stab})$, where $S$ is the $\nZ_2$-graded vector space
\be
S = \bigwedge( k\theta_1 \oplus \cdots \oplus k \theta_n )
\ee
with grading determined by $|\theta_i| = 1$. We make the tensor product $S \otimes_k \End_R(k^{\stab})$ into a DG-algebra in the usual way, giving $S$ zero differential.

\begin{definition} We define the finite-dimensional $\nZ_2$-graded vector space
\be\label{eq:defnunderlineend}
\underline{\End}(k^{\stab}) = R/\mf{m} \otimes_k \End_R(k^{\stab}) \cong \End_k\Big( \bigwedge( k\psi_1 \oplus \cdots \oplus k \psi_n) \Big)
\ee
where $\mf{m} = (x_1,\ldots,x_n)$.
\end{definition}

We will show that $S \otimes_k \End_R(k^{\stab})$ is homotopy equivalent to $\underline{\End}(k^{\stab})$ viewed as a complex with zero differential. The corresponding homotopy retract (see Proposition \ref{prop:he_start} below) is our starting point for the minimal model construction. For this we use the Koszul complex of $x_1,\ldots,x_n$,
\be\label{eq:defnkoszulK}
K = \Big( R \otimes_k \bigwedge( k\theta_1 \oplus \cdots \oplus k \theta_n ), \; d_K = \sum_i x_i \theta_i^*\; \Big)
\ee
and the following diagram of $\nZ_2$-graded complexes and homotopy equivalences over $k$:
\be\label{eq:first_he}
\xymatrix@C+3pc{
S \otimes_k \End_R(k^{\stab}) \ar@<1ex>[r]^-{\exp(-\delta)} & K \otimes_R \End_R(k^{\stab}) \ar@<1ex>[l]^-{ \exp(-\delta) } \ar@<1ex>[r]^-{\pi} & \underline{\End}(k^{\stab})\,.\ar@<1ex>[l]^-{ \sigma_\infty }
}
\ee
The notation is as follows:
\begin{itemize}
\item The complexes involved here are
\begin{gather*}
\big( S \otimes_k \End_R(k^{\stab}), \; \partial = [-,d_{k^{\stab}}] \,\big)\\
\big( K \otimes_R \End_R(k^{\stab}), \; d_K + \partial\, \big)\\
\big( \underline{\End}(k^{\stab}), 0 \big)\,.
\end{gather*}
\item The operator $\psi_j = \psi_j \wedge -$ on $k^{\stab}$ satisfies
\[
[ \psi_j, d_{k^{\stab}} ] = \big[ \psi_j, \sum_i x_i \psi_i^* \big] = x_j \cdot 1\,,
\]
where $[-,-]$ always denotes the graded commutator. That is, $\psi_j$ is a homotopy for the action of $x_j$ on $k^{\stab}$. It is then easy to check that the odd operator $\alpha \mapsto \psi_j \circ \alpha$ on $\End_R(k^{\stab})$ is also a homotopy for $x_j$, and where it will not cause confusion we will also write $\psi_j$ for this operator of post-composition.

\item Both $S \otimes_k \End_R(k^{\stab})$ and $K \otimes_R \End_R(k^{\stab})$ have the same underlying $\nZ_2$-graded $k$-module, namely
\[
R \otimes_k \bigwedge( k\theta_1 \oplus \cdots \oplus k \theta_n ) \otimes_k \End_k\big( \bigwedge( k\psi_1 \oplus \cdots \oplus k \psi_n ) \big)\,.
\]
On this space we define the even operator
\be
\delta = \sum_{i=1}^n \psi_i \theta_i^*\,,
\ee
where $\psi_i$ is the operator on $\End_R(k^{\stab})$ discussed above, given by $\alpha \mapsto (\psi_i \wedge -) \circ \alpha$, and $\theta_i^*$ acts by contraction on $S$. It is easy to check that $\exp(\delta),\exp(-\delta)$ intertwines the differentials $\partial$ and $d_K + \partial$ and therefore gives an isomorphism between the first two complexes in \eqref{eq:first_he} \cite[Proposition 4.11]{murfet}.
\item The morphism of complexes $\pi: K \lto R/\mf{m}$ is defined by composing the projection of $K$ onto the submodule $R \cdot 1$ of $\theta$-degree zero forms, with the quotient $R \lto R/\mf{m}$. We also write $\pi$ for the result of tensoring this morphism with $\End_R(k^{\stab})$ to obtain
\[
\xymatrix@C+2pc{
K \otimes_R \End_R(k^{\stab}) \ar[r]^-{\pi \otimes 1} & R/\mf{m} \otimes_R \End_R(k^{\stab}) = \underline{\End}(k^{\stab})\,.
}
\]
This is a morphism of complexes, because the differential on $\End_R(k^{\stab})$ vanishes on the quotient $\underline{\End}(k^{\stab})$ by the hypothesis that $W \in \mf{m}^2$.
\item The $k$-linear homotopy inverse $\sigma_\infty$ to $\pi$ in \eqref{eq:first_he} is defined in terms of the following ingredients. The first is the $k$-linear connection (viewing $\theta_i$ as a $1$-form)
\be
\Delta: K \lto K\,, \qquad \Delta = \sum_i \partial_{x_i} \theta_i
\ee
and the degree $-1$ (with respect to the $\theta$-degree) $k$-linear operator
\be
H = [d_K, \Delta]^{-1} \Delta\,.
\ee
We write $\partial$ for the differential on $\End_R(k^{\stab})$, and $\sigma: k \lto K$ for the map which sends $\lambda \in k$ to $\lambda \cdot 1 \in K$, and define $k$-linear maps
\begin{align}
\sigma_\infty = \sum_{m \ge 0} (-1)^m (H \partial)^m \sigma\,,\\
H_\infty = \sum_{m \ge 0} (-1)^m (H \partial)^m H\,.
\end{align}
Here $H_\infty$ is an odd operator on $K \otimes_R \End_R(k^{\stab})$. These maps satisfy:
\begin{align}
\pi \sigma_\infty &= 1\,,\\
 \sigma_\infty \pi &= 1 - [d_K + \partial, H_\infty]\,.
\end{align}
\end{itemize}

By inspection of \eqref{eq:first_he}, we therefore have the following:

\begin{proposition}\label{prop:he_start} There is a diagram of morphisms of $\nZ_2$-graded $k$-complexes
\be
\xymatrix@C+5pc{
\big( S \otimes_k \End_R(k^{\stab}), \;\partial\, \big) \ar@<1ex>[r]^-{\Phi} & \big( \underline{\End}(k^{\stab}), 0 \big)\,.\ar@<1ex>[l]^-{\Phi^{-1}}
}
\ee
where $\Phi = \pi \exp(-\delta), \Phi^{-1} = \exp(\delta) \sigma_\infty$ and $\widehat{H} = \exp(\delta) H_\infty \exp(-\delta)$ satisfy
\be
\Phi \Phi^{-1} = 1\,, \qquad \Phi^{-1} \Phi = 1 - [\partial, \widehat{H}]\,.
\ee
Thus, $\Phi$ and $\Phi^{-1}$ are mutually inverse homotopy equivalences over $k$.
\end{proposition}

\section{The minimal model}

We now take the homotopy retract from Proposition \ref{prop:he_start} as the input to the usual algorithm for the construction of an $A_\infty$-minimal model of the DG-algebra $S \otimes \End_R(k^{\stab})$.

\newpage


\section{Examples}

To describe the higher multiplications it is convenient to write $[ \psi_i, - ]$ for the natural operator on $\md{A}$ (giving $\psi_i, \psi_j^*$ the usual anticommutation relations for fermionic creation and annhiliation operators), that is
\begin{align*}
[ \psi_i, \psi_{j_1}^* \cdots \psi_{j_t}^* ] &= \sum_{l=1}^t (-1)^{l-1} \psi_{j_1}^* \cdots [\psi_i, \psi_{j_l}^* ] \cdots psi_{j_t}^*\\
&= \sum_{l=1}^t (-1)^{l-1} \delta_{i=j_l} \psi_{j_1}^* \cdots \psi_{j_t}^*\,.
\end{align*}
A general fact is that the higher multiplications on $\md{A}$ are linear combinations of products of such operators. For example, when $n = 2$ a standard term in $r_3$ would look like
\[
\Phi_0 \otimes \Phi_1 \otimes \Phi_2 \mapsto \lambda \cdot [ \psi_1, [ \psi_2, \Phi_0 ] ] \cdot [ \psi_1, \Phi_1 ] \cdot [ \psi_2, \Phi_2 ]\,.
\]
The coefficient $\lambda$ is computed as a Feynman amplitude on a binary planar tree with with incoming fermion states $\psi_1 \psi_2, \psi_1, \psi_2$ in the three leaves (respectively) and a simple list of allowed interactions, the most significant of which is a trivalent interaction vertex with an incoming fermion $\psi_j$ and outgoing fermion $\theta_i$ and bosons $\partial_{x_i}( x^\gamma )$ whenever the polynomial $W^j$ has a nonzero coefficient for the monomial $x^\gamma = x_1^{\gamma_1} x_2^{\gamma_2}$ (the fermions $\theta_i$ are the auxiliary spinor representation generators as in \cite{murfet}). There are two other interactions which do not depend on $W$, and which take place only at internal edges or internal vertices (respectively).

\begin{example} Let $W = x^d$ so that $\md{A} = k \oplus k \psi^*$. Then only $r_2$ and $r_d$ are nonzero and on $(\md{A}[1])^{\otimes d}$ the only basis element with a nonzero value under $r_d$ is
\[
r_d( \psi^* \otimes \cdots \otimes \psi^* ) = 1\,.
\]
Another way to say this is
\[
r_d( \Phi_0 \otimes \cdots \otimes \Phi_{d-1} ) = \prod_{i=0}^{d-1} [ \psi, \Phi_i ]\,.
\]
This $A_\infty$-structure is cyclic with respect to the trace form on $\md{A}$ which projects onto the $k \psi^*$-summand. Note that all such products are expanded such that the index $i$ increases from left to right.
\end{example}

\bibliographystyle{amsalpha}
\providecommand{\bysame}{\leavevmode\hbox to3em{\hrulefill}\thinspace}
\providecommand{\href}[2]{#2}
\begin{thebibliography}{BHLS03}
  
\bibitem{lazaroiu}
C.~I.~Lazaroiu, \textsl{Generating the superpotential on a D-brane category: I}, [arXiv:hep-th/0610120].
  
\bibitem{murfet}
D.~Murfet, \textsl{Computing with cut systems}, \href{http://arxiv.org/abs/1402.4541}{[arXiv:1402.4541]}.

\bibitem{lgdual}
N.~Carqueville and D.~Murfet, \textsl{Adjunctions and defects in Landau-Ginzburg models}, Adv. Math. \textbf{289} (2016), 480--566.

\bibitem{dm1102.2957}
T.~Dyckerhoff and D.~Murfet, \textsl{Pushing forward matrix factorisations}, Duke Math. J. \textbf{162} (2013), 1249--1311.

\end{thebibliography}

\end{document}